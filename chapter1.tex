% +--------------------------------------------------------------------+
% | Sample Chapter 1
% |
% | This file provides examples of how to
% | - insert a figure with a caption
% | - construct a table with a caption
% | - create subsections within the chapter
% | - insert a reference to a Figure or Table
% | - make a citation
% +--------------------------------------------------------------------+

\cleardoublepage

% +--------------------------------------------------------------------+
% | Replace "Chapter Title" below with the title of your chapter.
% | LaTeX will automatically number the chapters.
% +--------------------------------------------------------------------+

\chapter{Introduction}
\label{makereference1}

% Introduction to Thesis
%Current research indicates a growing need for laboratory components for introductory control theory classes. However, many hurdles like budget, class size, and space limitations arise when laboratories are appended to lectures in universities \citep{2,Experimential_Learning}. The \ac{NERMLAB} aims to address these concerns in reducing the overall cost imposed on instructors and students, as well as, minimizing the foot print of the hardware to allow students to take part in laboratories in a home environment. It is this home experimentation that allows students to engage in experimental learning.

%This thesis will attempt to address the feasability of the cheaper NERMLAB alternative. Multiple experiments will be conducted as they appear in Appendix \ref{Appendix:Key3} and results will be compared to more expensive hardware, such as the Motorlab.

%\subsubsection{Experimental Learning}
% Experimental Learning
%Experimental learning is a methodology that aims at creating knowledge through wisdom, observation, and insight from experience. Experimental learning also provides an alternative learning mechanism for the traditional theoretical components that make up a standard engineering curriculum. Most experimental learning is achieved through a laboratory practicum that helps students connect the theoretical ideas developed in lecture with what is done in practice. As a result, students can gain further insight into the theory that might have gone unresolved without experimental learning \citep{Experimential_Learning}. The goal is to allow students access to these experiences outside a classroom or laboratory environment.

% Connect these two subsubsections somehow

%\subsubsection{Laboratory Cost and Portability}
% Cheaper Hardware
%Classroom sizes continue to grow in universities, and, a direct result of this, is increasing laboratory size. Since size means the cost per student goes up due to the limited amount of equipment available, there is a desire for more afforable hardware \citep{4}. A way to combat the issue would be to make laboratory hardware more portable, allowing for cheaper components, such as motors, motor drivers and the like to be used \citep{Experimential_Learning}. However, utilzing cost-effective hardware in laboratory equipment can lead to poorly produced data, which does not adhere to theoretical models developed in lecture. While it is true that cheaper hardware does lead to less than desired data, it does allow greater access to students because of its cost. It is this aspect of portability that is of importance because students will be allowed to learn at their own pace, rather than having to adhere to strict laboratory procedures and times, by allowing students to take home the equipment and learn in a way that benefits them the most. Even when students do run experiments at home they still achieve the same learning objective as that of a traditional on-campus laboratory \citep{Experimential_Learning}.

%\subsubsection{Thesis Partitioning}


%%%%%%%%%%%%%%%%%%%%%%%%%%%%%%%%%%
Current research indicates a growing need for laboratory components for introductory control theory classes. However, many hurdles like budget, class size, and space limitations arise when laboratories are appended to lectures in universities \citep{2,Experimential_Learning}. The \ac{NERMLAB} aims to address these concerns in reducing the overall cost imposed on instructors and students, as well as, minimizing the foot print of the hardware to allow students to take part in laboratories in a home environment. It is this home experimentation that allows students to engage in experimental learning, which is a methodology that aims at creating knowledge through wisdom, observation, and insight from experience. Experimental learning also provides an alternative learning mechanism for the traditional theoretical components that make up a standard engineering curriculum. Most experimental learning is achieved through a laboratory practicum that helps students connect the theoretical ideas developed in lecture with what is done in practice. As a result, students can gain further insight into the theory that might have gone unresolved without experimental learning \citep{Experimential_Learning}. 

Unfortunately, classroom sizes continue to grow in universities, and, a direct result of this, is increasing laboratory size. Since size means the cost per student increases due to the limited amount of equipment available, there is a desire for more afforable hardware \citep{4}. A way to combat the issue would be to make laboratory hardware more portable, allowing for cheaper components, such as motors, motor drivers and the like to be used \citep{Experimential_Learning}. However, utilzing cost-effective hardware in laboratory equipment can lead to poorly produced data, which does not adhere to theoretical models developed in lecture. While it is true that cheaper hardware does lead to less than desired data, it does allow greater access to students because of its cost. The goal of NERMLAB is to give students access to affordable equipment that can provide them with the experimental learning opportunities both in the classroom and at home. In addition, it is this aspect of portability that is of importance because students will be allowed to learn at their own pace, in a way that benefits them the most, and still achieve the same learning objective as that of a traditional on-campus laboratories \citep{Experimential_Learning}.

This thesis will attempt to address the feasability of the cheaper NERMLAB alternative. Multiple experiments will be conducted as they appear in Appendix \ref{Appendix:Key3} and results will be compared to more expensive hardware, such as the Motorlab. Chapter \ref{chp2} will describe the NERMLAB system apparatus and the various components that comprise it, as well as, comment on the differences between the NERMLAB and the older Motorlab system. Then, Chapter \ref{chp3} will discuss system identification and characterization, which produces things such as the motor torque constant, inductance, and resistance. Chapter \ref{chp4} will then develop the necessary mathematical models that are necessary for the experiments that make up chapters \ref{chp5}-\ref{chp7}. 
% +--------------------------------------------------------------------+
% | Sample Chapter 1
% |
% | This file provides examples of how to
% | - insert a figure with a caption
% | - construct a table with a caption
% | - create subsections within the chapter
% | - insert a reference to a Figure or Table
% | - make a citation
% +--------------------------------------------------------------------+

\cleardoublepage

% +--------------------------------------------------------------------+
% | Replace "Chapter Title" below with the title of your chapter.
% | LaTeX will automatically number the chapters.
% +--------------------------------------------------------------------+

\chapter{Introduction}
\label{makereference1}

Current research indicates a growing need for laboratory components for introductory control theory classes. However, many hurdles like budget, class size, and space limitations arise when laboratories are added to lecture components in universities ~\citep{2}. This thesis will address some of these issues, like budget of laboratory hardware and space limitations, and try to assess the feasibility of utilizing low budget, smaller, and portable laboratory hardware for introductory control classes. The introduction of this thesis will be broken up into sections to address these issues individually. It's important to note, that while this thesis does recognize the importance of having laboratory components to control theory lectures, it will not be the main focus, rather importance will be given to addressing the feasibility of low budget portable hardware, more specifically the Motorlab, for control applications. This chapter serves more as a background to why lower budget, portable hardware is important to control classes. 

\section{Hardware and Software Budget}
\label{makereference1.1}

\section{Space Limitations}
\label{makereference1.2}
% +--------------------------------------------------------------------+
% | Abstract Page
% +--------------------------------------------------------------------+

\pagestyle{empty}
%\vspace{1cm}
\setlength{\baselineskip}{0.8cm}

%\indent

% +--------------------------------------------------------------------+
% | Enter the text of your abstract below, maximum of 500 words.
% +--------------------------------------------------------------------+

Control theory is a methodology investigated by many mechanical and electrical engineering students throughout most universities in the world. Because of control theory?s broad and interdisciplinary nature, it necessitates further study by application through laboratory practice. Typically the hardware used to connect the theoretical aspects of controls to the practical can be expensive, big, and time consuming to the students and instructors teaching on the equipment. This is due to the fact that connecting various hardware components such as sensors, encoders, amplifiers, and motors can lead to data that does not fit perfectly the theoretical mold developed in the controls classroom, further dissuading students of the idea that there exists a connection between developed theoretical models and what is seen in practice. 

There is a recent trend in universities wishing to develop open-source, inexpensive hardware for various applications. This thesis will investigate and conduct a multitude of experiments on an apparatus known as the Motorlab to determine the feasibility of such equipment in the field of control theory application. The results will be compared against time-tested hardware to demonstrate the practicality of open-source, inexpensive hardware.

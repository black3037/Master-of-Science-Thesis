% +--------------------------------------------------------------------+
% | Sample Chapter 3
% +--------------------------------------------------------------------+

\cleardoublepage

% +--------------------------------------------------------------------+
% | Replace "This is Chapter 3" below with the title of your chapter.
% | LaTeX will automatically number the chapters.                      
% +--------------------------------------------------------------------+

\chapter{Model Development}
\label{chp3}

Chapter \ref{chp3} will be dedicated to developing the various parameters that make up the NERMLAB such as the motor torque constant, \ac{back-emf}, inductance, and max voltage. Each section in Chapter \ref{chp3} will detail the process of how the various parameters were measured, calculated, and experimentally determined. Nomenclature for various constants and parameters are detailed in the table \ref{table2}.

% Table of motor lab constants
\begin{table}[ht]
\begin{center}
\caption{Motor parameters}
\begin{tabular}[c]{|c|c|}

\hline
\textbf{Parameter} & \textbf{Description}\\

\hline
V & Motor Voltage\\

\hline
\(k_t\) & Motor Torque Constant\\

\hline
\(K_E\) & Back Electromotive Force Constant\\

\hline
J & Mass Moment of Inertia\\

\hline
L & Motor Inductance\\

\hline
R & Motor Resistance\\

\hline
\(\tau\) & Time Constant\\

\hline
\end{tabular}

\label{table2}
\end{center}
\end{table}

% end of table

\section{Motor Resistance}

\section{Motor Torque Constant and Back EMF}
In order to determine the Motor Torque Constant (\(k_t\)), the back-emf needs to be determined in order to make an appropriate estimate for \(k_t\). Detail experiment here.

\[K_E = \frac{V}{\omega_m}\]


\section{Mass Moment of Inertia (\(J\)) Estimation}

\section{Motor Inductance}

\[L = R\tau\]